\documentclass{ees}

\shorttitle{Missa Quia mihi}

\begin{document}

\eesTitlePage

\eesCriticalReport{
  1 & –     & chorus & \B{1-a} (which presumably represents the earlier source)
                       contains durations and beams that support the
                       hyphenation “e–le–i–son”, while \B{1-b} (S, A),
                       \B{1-c} (T, B), and \B2 rather favor the
                       hyphenation “e–lei–son”.
                       Here, the latter variant is preferred. \\
    & 1–8     & vl 1 & In \B{1-c}, vl 1 originally played unison with ob 1.
                       This was later changed to the shown version.
                       In \B2, vl 2 plays unison with ob 1. \\
    & 3       & A    & bar in \B2: a′1 \\
    & 3       & T    & bar in \B2: \sharp f′1 \\
    & 3       & B    & bar in \B2: \sharp d1 \\
    & 7       & vl 2 & bar in \B2: b′1 \\
    & 7       & S    & 2nd \halfNote\ in \B{1-a}: \sharp g′4.–\sharp g′8 \\
    & 7       & A    & bar in \B{1-a}: e′2.–e′4 \\
    & 7       & T    & bar in \B{1-a}: b′2.–b′4 \\
    & 7       & B    & bar in \B{1-a}: e2.–e4 \\
    & 9–16    & vl   & violins pause in \B2 \\
    & 12      & A    & 1st \halfNote\ in \B{1-a}: \sharp f′4.–\sharp f′8 \\
    & 16      & T    & 1st \halfNote\ in \B{1-a} and \B{1-c}:
                       \sharp g4.–\sharp g8 \\
    & 16      & B    & 2nd \quarterNote\ in \B{1-a}: \sharp c8.–\sharp c16 \\
    & 21      & vla  & 6th \eighthNote\ in \B{1-d1} and \B{1-d2}: d8 \\
    & 24      & B    & 1st \halfNote\ in \B{1-a} and \B{1-c}:
                       \sharp f4.–\sharp f8 \\
    & 25      & T    & 1st \halfNote\ in \B{1-a} and \B{1-c}:
                       \sharp g4.–\sharp g8 \\
    & 25      & B    & 1st \halfNote\ in \B{1-a} and \B{1-c}:
                       \sharp G4.–\sharp G8 \\
    & 34      & ob 1 & 1st \quarterNote\ in \B{1-b}: b′8–b′8 \\
    & 30–41   & vl 2 & in \B2 unison with S \\
    & 33–41   & vl 1 & in \B2 unison with S \\
    & 40f     & S    & 2nd \halfNote\ of bar 40 and bar 41 in \B{1-a}:
                       b′4.–b′8 and \sharp c″2–\sharp c″2 \\
    & 40      & A    & bar in \B{1-a}: e′2.–e′4 \\
    & 40      & B    & bar in \B{1-a}: e2.–e4 \\
  \midrule
  2 & 12      & org  & 1st \halfNote\ in \B2: a8–\sharp g8–\sharp f8–e8 \\
    & 31–38   & vl   & instruments pause in \B2 \\
    & 45      & vl 2 & 4th \quarterNote\ in \B{1-c} and \B2: \sharp f′4 \\
    & 47      & vla  & bar in \B{1-c} and \B{1-d}: b4–b4–b4.–b8 \\
    & 48      & vl 2 & 1st \quarterNote\ in \B2: \sharp e′4 \\
    & 61–111  & vl 2 & in \B2 unison with vl 1 and S \\
    & 66      & org  & upper voice in \B{1-b} and \B{1-f}: b′1. \\
    & 68      & org  & lower voice in \B{1-b} and \B{1-f}: d′1. \\
    & 93      & vla  & bar in \B{1-c} and \B{1-d}: a′2–\sharp g′1 \\
    & 93      & T    & bar in \B{1-a}, \B{1-c}, and \B2: a′2–\sharp g′1 \\
    & 108     & A    & bar in \B{1-a}: e′2–e′1 \\
    & 114     & vl 2 & 2nd \quarterNote\ in \B2: \sharp f″4 \\
    & 117     & vl 2 & 2nd \quarterNote\ in \B2: b′4 \\
    & 125     & vl 2 & last \quarterNote\ in \B2:
                       e′16–\sharp f′16–\sharp g′16 \\
    & 128     & vl 1 & 6th \eighthNote\ in \B2: a8 \\
    & 137–163 & vl 1 & In \B{1-c}, vl 1 originally played one octave lower.
                       This was later changed to the shown version.
                       In \B2, vl 1 plays unison with S. \\
    & 147     & A    & bar in \B2: a1 \\
    & 155     & T    & last \quarterNote\ in \B2: b4 \\
    & 156     & vla  & 1st \halfNote\ in \B{1-d1}: \sharp c′4–b4 \\
    & 156     & T    & 2nd \halfNote\ in \B{1-a} and \B{1-c}: \sharp c′2 \\
    & 158     & org  & upper voice added by editor \\
    & 176     & A    & bar in \B{1-a}: \sharp f′1 \\
  \midrule
  3 & 1–27    & vl   & instruments pause in \B2 \\
    & 13      & org  & 2nd/3rd \quarterNote\ in \B2 (org): e2 \\
    & 18      & A    & bar in \B{1-a}: \sharp g′4.–\sharp f′8–\sharp f′4 \\
    & 49      & vl 2 & 3rd \quarterNote\ in \B{1-d} and \B2: e′4 \\
    & 71      & vl 1 & 2nd \quarterNote\ in \B1: c″4 \\
    & 100     & T    & last \quarterNote\ in \B{1-a}: \sharp a8.–\sharp a16 \\
    & 101     & vl 1 & 1st \quarterNote\ in \B1: b8–g″8 \\
    & 101     & vl 2 & 1st \quarterNote\ in \B1: b8–g″8 \\
    & 137     & B    & last \eighthNote\ in \B2: e8 \\
    & 145–147 & ob 2 & in \B{1-b} unison with vl 2, here emended\newline
                       to accomodate the oboe’s range \\
    & 152     & vl 2 & 1st \quarterNote\ in \B2: a′4 \\
    & 155–175 & vl 2 & in \B2 unison with vl 1 and S \\
  \midrule
  4 & 6       & vl 2 & 1st \quarterNote\ in \B{1-c} and \B2: b4 \\
    & 8       & vla  & 3rd \eighthNote\ in \B{1-c} and \B{1-d}: d′8 \\
    & 10–30   & vl 2 & in \B2 unison with vl 1 and S \\
    & 15      & S    & 2nd/3rd \halfNote\ in \B{1-a}:
                       \sharp d″2–\sharp c″4–\sharp d″4 \\
    & 26      & vl 1, S & last \quarterNote\ in \B2: \sharp f′4 \\
  \midrule
  5 & 1       & vla  & last \quarterNote\ in \B{1-d}: \sharp g′4 \\*
    & 19–45   & vl 2 & in \B2 unison with vl 1 and S \\
}

\eesToc{}

\eesScore

\end{document}
