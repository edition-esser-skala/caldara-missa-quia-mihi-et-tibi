\documentclass{ees}

\begin{document}

\eesTitlePage

\eesCriticalReport{
  1 & –    & chorus & \B{1-a} (which presumably represents the earlier source)
                      contains durations and beams that support the
                      hyphenation “e–le–i–son”, while \B{1-b} (S, A) and
                      \B{1-c} (T, B) rather favor the hyphenation “e–lei–son”.
                      Here, the latter variant is preferred. \\
    & 1–8  & vl 1   & In \B{1-c}, vl 1 originally played unison with ob 1.
                      This was later changed to the version shown here. \\
    & 7    & S      & 2nd \halfNote\ in \B{1-a}: \sharp g′4.–\sharp g′8 \\
    & 7    & A      & bar in \B{1-a}: e′2.–e′4 \\
    & 7    & T      & bar in \B{1-a}: b′2.–b′4 \\
    & 7    & B      & bar in \B{1-a}: e2.–e4 \\
    & 12   & A      & 1st \halfNote\ in \B{1-a}: \sharp f′4.–\sharp f′8 \\
    & 16   & T      & 1st \halfNote\ in \B{1-a} and \b{1-c}:
                      \sharp g4.–\sharp g8 \\
    & 16   & B      & 2nd \quarterNote\ in \B{1-a}: \sharp c8.–\sharp c16 \\
    & 21   & vla    & 6th \eighthNote\ in \B{1-d1} and \B{1-d2}: d8 \\
    & 24   & B      & 1st \halfNote\ in \B{1-a} and \b{1-c}:
                      \sharp f4.–\sharp f8 \\
    & 25   & T      & 1st \halfNote\ in \B{1-a} and \b{1-c}:
                      \sharp g4.–\sharp g8 \\
    & 25   & B      & 1st \halfNote\ in \B{1-a} and \b{1-c}:
                      \sharp G4.–\sharp G8 \\
    & 34   & ob 1   & 1st \quarterNote\ in \B{1-b}: b′8–b′8 \\
    & 40f  & S      & 2nd \halfNote\ of bar 40 and bar 41 in \B{1-a}:
                      b′4.–b′8 and \sharp c″2–\sharp c″2 \\
    & 40   & A      & bar in \B{1-a}: e′2.–e′4 \\
    & 40   & B      & bar in \B{1-a}: e2.–e4 \\
  \midrule
  2 & 45   & vl 2   & 4th \quarterNote\ in \B{1-c}: \sharp f′4 \\
    & 47   & vla    & bar in \B{1-c} and \B{1-d}: b4–b4–b4.–b8 \\
}

\eesToc{}

\eesScore

\end{document}
